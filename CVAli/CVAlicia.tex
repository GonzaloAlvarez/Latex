%% start of file `template_en.tex'.
%% Copyright 2006-1008 Xavier Danaux (xdanaux@gmail.com).
%
% This work may be distributed and/or modified under the
% conditions of the LaTeX Project Public License version 1.3c,
% available at http://www.latex-project.org/lppl/.


\documentclass[11pt,a4paper]{moderncv}

% moderncv themes
%\moderncvtheme[blue]{casual}                 % optional argument are 'blue' (default), 'orange', 'red', 'green', 'grey' and 'roman' (for roman fonts, instead of sans serif fonts)
%\moderncvtheme[blue]{casualg}                 % optional argument are 'blue' (default), 'orange', 'red', 'green', 'grey' and 'roman' (for roman fonts, instead of sans serif fonts)
\moderncvtheme[blue]{casualg}                % idem
%\moderncvtheme[roman]{classic}                % idem

% character encoding
\usepackage[utf8]{inputenc}                   % replace by the encoding you are using
\usepackage{hyperref}
\usepackage{fmtcount}

% adjust the page margins
\usepackage[scale=0.80]{geometry}
%\setlength{\hintscolumnwidth}{3cm}						% if you want to change the width of the column with the dates
\AtBeginDocument{\setlength{\maketitlenamewidth}{10cm}}  % only for the classic theme, if you want to change the width of your name placeholder (to leave more space for your address details
\AtBeginDocument{\recomputelengths}                     % required when changes are made to page layout lengths

% personal data
\firstname{Alicia}
\familyname{Minguez}
\title{Curriculum Vitae}               % optional, remove the line if not wanted
\address{2707 Queen Anne Ave N}{Seattle WA 98109}    % optional, remove the line if not wanted
\mobile{206 883 7299}
% \fax{fax (optional)}                          % optional, remove the line if not wanted
\email{Alicia.Minguez@gmail.com}                      % optional, remove the line if not wanted
\extrainfo{Born in Madrid, 08 Sep 1982} % optional, remove the line if not wanted
%\photo[64pt]{picture.jpg}                         % '64pt' is the height the picture must be resized to and 'picture' is the name of the picture file; optional, remove the line if not wanted
% \quote{}                 % optional, remove the line if not wanted

\nopagenumbers{}                             % uncomment to suppress automatic page numbering for CVs longer than one page
\newenvironment{pitems}{
	\begin{itemize}
	\setlength{\itemsep}{1pt}
	\setlength{\parskip}{0pt}
	\setlength{\parsep}{0pt}
}{\end{itemize}}


%----------------------------------------------------------------------------------
%            content
%----------------------------------------------------------------------------------
\begin{document}
\maketitle

\section{Education}
\cventry{2004--2007}{Master's Degree in Electrical Engineering and \mbox{Information Technology}}{\mbox{Technische} Universitaet Darmstadt}
	{Darmstadt, Germany}{Double degree Spain/Germany}{}
\cventry{2001--2007}{Master's Degree in Telecommunications Engineering}{Universidad Politecnica de Madrid}{Madrid, Spain}{}{}

\section{Work Experience}
\ccventry{2011--Present}{Senior Systems and Validation Engineer for Air Traffic Management}{Indra}{Madrid, Spain}
	{. Activities:
\begin{pitems}
\item $\bullet$ \textsl{Software Validation Engineer:} Definition and implementation of validation tests plans for the 
	Air Traffic Control Software for several national corporations (NATS, DFS). Requirements at	subsystem level are tested against the real 
	Traffic Control infrastructure in order to assure the viability and stability of the software elements. 
	Definition and recollection of recommendations for the enhancement of the software elements that help improve the quality of the final systems.
\item $\bullet$ \textsl{Systems Engineer:} Definition and management of full product lifecycle for the Air Traffic Control Software
	ensuring the completion of the goals required by the client.
\item $\bullet$ \textsl{Network Inspection Controller:} Network validation tasks in order to ensure the proper information
	exchange at network level. Use of several networking analysis tools (like wireshark) and generation of technical reports based on the results.
\item $\bullet$ \textsl{Technical Documentation Reviewer:} Definition and preparation of the technical data used to
	generate the proper documentation following strict documentation models and standards used in the Air Traffic Control industry
\end{pitems}}

\ccventry{2009--2011}{Ground Station Systems Engineer at ESOC/ESA}{Serco contractor on site at ESOC/ESA OPS-GSY section}{Darmstadt, Germany}
	{. Activities:
\begin{pitems}
\item $\bullet$ \textsl{Validation Engineer:} Definition and implementation of end-to-end testing for ground segment. 
		Cooperation with subsystem developers and ground station \mbox{engineers reviewing} and validating the tailoring specifications, checking the site 
		\mbox{acceptance} software validation plans.	
\item $\bullet$ \textsl{System Engineer:} Maintenance and operation of the ground segment reference \mbox{facilities} located in ESOC. 
		Management of hardware providers for the equipment used in the Ground Segment Reference Facilities.
\item $\bullet$ \textsl{Signal Engineer:} Engineer for Delta DOR (Differential One-way Ranging) used for interplanetary navigation.
		Collaboration in the implementation of radio frequency compatibility tests for the ground segment and spacecraft equipment.
\end{pitems}}
\cventry{2009--2009}{Antenna Software Engineer}{Inform GmbH contractor at MT Mechatronics GmbH}{Mainz, Germany}
{Construction and comissioning of a new antenna for the SAR-Lupe Gelsdorf project. 
		Design and development of local control panel for the antenna. 
		\mbox{Development} and validation of real time software for the Antenna Control Unit}{}{}

\cventry{2008--2008}{Research Engineer}{NEC Laboratories Europe}{Heidelberg, Germany}
	{Web services development of IP multimedia subsystem based IPTV applications for the project "IPTV Value Added Services"}{}{}

\cventry{2007--2008}{Young Graduate Trainee}{ESOC/ESA OPS-GDS section}{Darmstadt, Germany}
	{Definition and analysis of software requirements for Gaia spacecraft simulator. 
		Collaboration in a generic software requirements specification document for spacecraft simulators. 
		Definition of requirements and support for Lisa Pathfinder spacecraft simulator}{}{}

\cventry{2006--2007}{Diploma Thesis}{Honda Research Institute Europe GmbH}{Offenbach, Germany}
	{\mbox{Digital} image processing for the visual system of Honda's robot Asimo}{}{}

\cventry{2005--2006}{Student Position}{Deutsche Telekom AG}{Darmstadt, Germany}{Development and testing of central identity management system for IT-Security}{}{}

\section{Languages}
\cvlanguage{Spanish}{Mother Tongue}{}
\cvlanguage{English}{Fluent}{Several years working in English in different companies.}
\cvlanguage{German}{Fluent}{Seven years living and working in Germany.}
\cvlanguage{French}{Basic}{}

\section{Professional Skills}
\cvskill{Programming}{C++, IEC 61131-3, Java, Matlab, VB, JavaScript}
\cvskill{Tools}{DOORS, Eclipse, TwinCAT, MS-Office}
\cvskill{Technologies}{PLC Programming, MFC, Web Services, UML, XML, SQL}

\end{document}


%% end of file `template_en.tex'.
