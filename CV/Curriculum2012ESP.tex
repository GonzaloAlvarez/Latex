%% start of file `template_en.tex'.
%% Copyright 2006-1008 Xavier Danaux (xdanaux@gmail.com).
%
% This work may be distributed and/or modified under the
% conditions of the LaTeX Project Public License version 1.3c,
% available at http://www.latex-project.org/lppl/.


\documentclass[11pt,a4paper]{moderncv}

% moderncv themes
%\moderncvtheme[blue]{casual}                 % optional argument are 'blue' (default), 'orange', 'red', 'green', 'grey' and 'roman' (for roman fonts, instead of sans serif fonts)
%\moderncvtheme[blue]{casualg}                 % optional argument are 'blue' (default), 'orange', 'red', 'green', 'grey' and 'roman' (for roman fonts, instead of sans serif fonts)
\moderncvtheme[blue]{classicg}                % idem
%\moderncvtheme[roman]{classic}                % idem

% character encoding
\usepackage[utf8]{inputenc}                   % replace by the encoding you are using
\usepackage{hyperref}

% adjust the page margins
\usepackage[scale=0.80]{geometry}
%\setlength{\hintscolumnwidth}{3cm}						% if you want to change the width of the column with the dates
\AtBeginDocument{\setlength{\maketitlenamewidth}{10cm}}  % only for the classic theme, if you want to change the width of your name placeholder (to leave more space for your address details
\AtBeginDocument{\recomputelengths}                     % required when changes are made to page layout lengths

% personal data
\firstname{Gonzalo}
\familyname{Alvarez}
\title{Curriculum Vitae}               % optional, remove the line if not wanted
\address{C/ Ponzano 80, 6 6}{28003 Madrid}    % optional, remove the line if not wanted
\mobile{0034 678 252 458}                      % optional, remove the line if not wanted
% \fax{fax (optional)}                          % optional, remove the line if not wanted
\email{gonzaloab@gmail.com}                      % optional, remove the line if not wanted
\website{www.gonzaloalvarez.es}
\extrainfo{Le\'on (Espa\~na), 12.31.81} % optional, remove the line if not wanted
%\photo[64pt]{picture.jpg}                         % '64pt' is the height the picture must be resized to and 'picture' is the name of the picture file; optional, remove the line if not wanted
% \quote{}                 % optional, remove the line if not wanted

\nopagenumbers{}                             % uncomment to suppress automatic page numbering for CVs longer than one page


\newenvironment{pitems}{
	\begin{itemize}
	\setlength{\itemsep}{1pt}
	\setlength{\parskip}{0pt}
	\setlength{\parsep}{0pt}
}{\end{itemize}}


%----------------------------------------------------------------------------------
%            content
%----------------------------------------------------------------------------------
\begin{document}
\maketitle

\section{Formaci\'on Acad\'emica}
\cventry{2000--2007}{Ingeniero Superior de Telecomunicaciones}{Universidad de Vigo}{}{}{}  % arguments 3 to 6 are optional
\cventry{2005--2007}{Proyecto Fin de Carrera}{Universidad de Vigo}{Desarrollo de un entorno Web en ASP y adaptaci\'on a la LOPD}{}{}{}

\section{Experiencia Laboral}
\ccventry{2011--Presente}{Ingeniero Senior de Software y Sistemas}{Consultor Senior para The Server Labs}{Madrid, Spain}
{. Tareas:
\begin{pitems}
\item $-$ \textsl{Ingeniero de Desarrollo en el Proyecto FARO:} Desarrollo de una Plataforma Web para el banco BBVA basada en J2EE.
	El Proyecto FARO es un Entorno Web desarrollado usando el conocido modelo Controlador-Servicio-DAO bas\'andose en el Framework Spring.
	Emplea JavaServer Faces y Primefaces para la capa de presentaci\'on, Java Persistence API para administrar las bases de datos relacionales
	y OracleDB como entorno de base de datos. Ha sido desarrollado usando metodolog\'ias Agile en un modelo de desarrollo iterativo e incremental.
\item $-$ \textsl{Arquitecto de Sistemas Cloud:} Dise\~no y mantenimiento de entornos Cloud para empresas basado en servicios de Amazon, como EC2,
	S3, AutoEscalado, ELB, ElasticIP, RDS, Route53. Desarrollo de herramienta de Provisioning para Cloud FastTrack en Java orientada a facilitar
	el despliegue de systemas en el Cloud.
\item $-$ \textsl{Sistema de Integraci\'on Continuo del Proyecto FARO:} Administraci\'on del entorno de CI usado por el equipo de desarrollo
	del proyecto FARO integrado por herramientas como Hudson, SVN o Jira, adem\'as de la administraci\'on de los sistemas de desarrollo y
	preproducci\'on.
\item $-$ \textsl{Consultor T\'ecnico:} Preparaci\'on de propuestas, ofertas y documentos t\'ecnicos.
\end{pitems}}
\ccventry{2009--2011}{Ingeniero de Software y Sistemas para la Agencia Espacial Europea (ESOC/ESA)}
{Contratista Serco para ESOC/ESA, Secci\'on OPS-GIB}{Darmstadt, Alemania}
{. Tareas:
\begin{pitems}
\item $-$ \textsl{Ingeniero de Validaci\'on:} Definici\'on, aplicaci\'on y documentaci\'on de procedimientos de validaci\'on de varios 
	sistemas desplegados en las Estaciones de Tierra de la red de ESA como FIDES (File Disseminator for ESTRACK Ground Stations) o 
	FARC (File Archive usado por SCOS para almacenar telemetr\'ia).
\item $-$ \textsl{Ingeniero de TMTCS:} Responsable t\'ecnico del mantenimiento del sistema de Telemetr\'ia y Telecomando. A cargo de
	la validaci\'on y la integraci\'on del sistema mediante la creaci\'on de informes de errores y cambios usando varias herramientas 
	como DOORS, Mercury Quality Center y Sinergy.
\item $-$ \textsl{Scrum Master e Ingeniero de Software:} Arquitecto y desarrollador de varias de las aplicaciones usadas en
	la red de Estaciones de Tierra de la ESA, tal como EUDGSSC (una interfaz gr\'afica para interactuar con el controlador
	gen\'erico de subsistemas) o EUD MIMICS (un interfaz mimic para dotar a las aplicaciones EGOS de un entorno visual).
\item $-$ \textsl{Administraci\'on de servidores:} Administrador jefe encargado de la gesti\'on del servidor principal de la secci\'on
	de infraestructura de la Agencia Espacial Europea usado para intercambio de documentaci\'on y gesti\'on de projectos de software.
\item $-$ \textsl{Asesor T\'ecnico:} Colaborador en comit\'es t\'ecnicos como asesor para numerosos proyectos como FEC, STC2, GSMC, etc.
\end{pitems}}
\ccventry{2008--2009}{YGT en ESOC/ESA}{Young Graduate Trainee en ESOC/ESA, Secci\'on OPG-GIB}{Darmstadt, Alemania}
{. Tareas:
\begin{pitems}
\item $-$ \textsl{Ingeniero de Validaci\'on:} Asesor t\'ecnico y supervisor de fase de pruebas FAT del EGOS User Desktop. Gestor
	t\'ecnico de la fase de pruebas SAT.
\item $-$ \textsl{Ingeniero de Software:} Revisi\'on t\'ecnica de los requisitos del proyecto MIMIC y desarrollador de un 
	prototipo de uso interno basado en Eclipse mediante el uso de las tecnolog\'ias EMF, GEF y GMF.
\end{pitems}}
\cventry{2007--2007}{Profesor de Project Management}{Seminario de Project Management impartidos a IMERYS IKF Inc.}
{A Guarda, Pontevedra}{Profesor de cursos de empresa para Norformacion - Fundaci\'on Tripartita a IMERYS IKF Inc}{}
\cventry{2005--2006}{Investigador Contratado}{Universidad de Vigo}{Desarrollo de JPoEML,
	una herramienta Java usada para crear entornos Web de tele ense\~nanza que usa XML y J2EE para servir
	programas de formaci\'on a trav\'es de la Red}{}{} 
\cventry{1999--2005}{Administrador de Redes y Desarrollador Web}{Desarrollador Web Freelance y administrador de redes y sistemas para 
	empresas de \'ambito nacional}{Vigo}{}{}

\section{Formaci\'on Complementaria}
\cventry{Abr 2010}{Agile Development Technologies}{The Server Labs}{ESOC, Darmstadt, Alemania}
{Desarrollo de software usando metodolog\'ias Agile y t\'ecnicas SCRUM}{}
\cventry{Feb 2009}{Eclipse EMF, GEF and GMF Technologies}{OBEO Eclipse Foundation Member}{ESOC, Darmstadt, Alemania}
{Desarrollo de aplicaciones Eclipse basadas en EMF/GEF y GMF}{}{}

\section{Idiomas}
\cvlanguage{Espa\~nol}{Lengua materna}{}
\cvlanguage{Ingl\'es}{Muy Alto}{Tres a\~nos trabajando en ingl\'es en la Agencia Espacial Europea. Seis meses viviendo en Toronto, Canada.}
\cvlanguage{Alem\'an}{Alto}{Tres a\~nos viviendo en Alemania, Goethe Institut nivel B2.}
\cvlanguage{Italiano}{Medio}{Cursos de formaci\'on en ESOC/ESA.}

\section{Desarrollo de Software}
\cvskill{Linear}{Assembler, C, Pascal, Cobol, Fortran, Lisp.}
\cvskill{Object Oriented}{Java, C++, Delphi, Visual Basic.}
\cvskill{Eclipse}{PDE, RCP, SWT, EMF, GEF, GMF, OCL.}
\cvskill{J2EE}{JSF, JPA, Hibernate, Richfaces.}
\cvskill{Distributed}{RCP, Java RMI, CORBA.}
\cvskill{Scripting}{Python, Bash, Tcl, Perl.}
\cvskill{Web}{PHP, ASP, JSP, GWT, DHTML, XML, CSS, JS, AJAX.}
\cvskill{CI}{Hudson, Jira, SVN, Trac, Maven, Ant.}
\cvskill{Agile}{Agilefant, IceScrum, XPlanner.}
\cvskill{Others}{SQL, \TeX.}

\section{Ingenier\'ia de Sistemas}
\cvline{Unix}{Administraci\'on de Sistemas Debian y RedHat Linux, Solaris / Irix. LAMP.}
\cvline{Windows}{Administraci\'on de Sistemas NT/XP/2003/7.}
\cvline{Redes}{HTTP, DNS, SMTP, FTP, Samba, DHCP, SSH, NFS, Security.}
\cvline{Virtualization}{Xen, OpenVZ, VirtualBox, VMWare.}
\cvline{Cloud}{Amazon EC2 S3 AutoScaling ELB Route53, Google Apps.}

\section{Documentaci\'on y Referencias}
\cvline{}{\slshape{Disponibles a petici\'on.}}

\end{document}


%% end of file `template_en.tex'.
