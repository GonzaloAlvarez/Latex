%% start of file `template_en.tex'.
%% Copyright 2006-1008 Xavier Danaux (xdanaux@gmail.com).
%
% This work may be distributed and/or modified under the
% conditions of the LaTeX Project Public License version 1.3c,
% available at http://www.latex-project.org/lppl/.


\documentclass[11pt,a4paper]{moderncv}

% moderncv themes
%\moderncvtheme[blue]{casual}                 % optional argument are 'blue' (default), 'orange', 'red', 'green', 'grey' and 'roman' (for roman fonts, instead of sans serif fonts)
%\moderncvtheme[blue]{casualg}                 % optional argument are 'blue' (default), 'orange', 'red', 'green', 'grey' and 'roman' (for roman fonts, instead of sans serif fonts)
\moderncvtheme[blue]{classicg}                % idem
%\moderncvtheme[roman]{classic}                % idem

% character encoding
\usepackage[utf8]{inputenc}                   % replace by the encoding you are using
\usepackage{hyperref}

% adjust the page margins
\usepackage[scale=0.8]{geometry}
%\setlength{\hintscolumnwidth}{3cm}						% if you want to change the width of the column with the dates
\AtBeginDocument{\setlength{\maketitlenamewidth}{10cm}}  % only for the classic theme, if you want to change the width of your name placeholder (to leave more space for your address details
\AtBeginDocument{\recomputelengths}                     % required when changes are made to page layout lengths

% personal data
\firstname{Gonzalo}
\familyname{Alvarez}
\title{Curriculum Vitae}               % optional, remove the line if not wanted
\address{C/ Ponzano 80, 6 6}{28003, Madrid}    % optional, remove the line if not wanted
\mobile{0034 678 252 458}                      % optional, remove the line if not wanted
% \fax{fax (optional)}                          % optional, remove the line if not wanted
\email{gonzaloab@gmail.com}                      % optional, remove the line if not wanted
\website{www.gonzaloalvarez.es}
\extrainfo{León (Spain), 12.31.81} % optional, remove the line if not wanted
%\photo[64pt]{picture.jpg}                         % '64pt' is the height the picture must be resized to and 'picture' is the name of the picture file; optional, remove the line if not wanted
% \quote{}                 % optional, remove the line if not wanted

\nopagenumbers{}                             % uncomment to suppress automatic page numbering for CVs longer than one page


\newenvironment{pitems}{
	\begin{itemize}
	\setlength{\itemsep}{1pt}
	\setlength{\parskip}{0pt}
	\setlength{\parsep}{0pt}
}{\end{itemize}}


%----------------------------------------------------------------------------------
%            content
%----------------------------------------------------------------------------------
\begin{document}
\maketitle

\section{Education}
\cventry{2000--2007}{Master's Degree in Telecommunications Engineering}{Vigo University}{}{}{}  % arguments 3 to 6 are optional
\cventry{2005--2007}{Master's Final Project}{Vigo University}{ASP Web Environment development}{}{}{}

\section{Work Experience}
\ccventry{2011--Today}{Senior System and Software Engineer}{Senior Consultant at The Server Labs}{Madrid, Spain}
{. Activities:
\begin{pitems}
\item $-$ \textsl{Software Engineer at FARO Project:} Development of a Web Platform for the BBVA Bank based on J2EE. The FARO Project
	is a Web Environment developed using the well known Controller-Service-DAO pattern based on the Spring Framework. It uses JavaServer Faces
   	and Primefaces for the presentation layer, Java Persistence API for managing the relational data and OracleDB as Database backend. It has been developed
	using Agile methodologies based on iterative and incremental development.
\item $-$ \textsl{Cloud Systems Architect:} Design and maintenance Cloud Environments based on Amazon Web Services, such as EC2,
	S3, AutoScaling, ELB, Elastic IP, Relational Database Services. Development of a Provisioning Tool for Cloud Fast Track
	in Java intended to ease the deployment of common configurations on new clouds.
\item $-$ \textsl{Continuous Integration Management at FARO Project:} Management of the CI environment
	used by the team developing the FARO project involving management of Hudson, SVN and Jira, along with the administration
	of the remote systems used for Continuous Deployment and Test Environments.
\item $-$ \textsl{Technical Consultant:} Participation in the preparation of internal documents and technical proposals.
\end{pitems}}
\ccventry{2009--2011}{System and Software Engineer at European Space Agency Operation Centre (ESOC/ESA)}{Serco Contractor on site at ESOC/ESA OPS-GIB Section}{Darmstadt, Germany}
{. Activities:
\begin{pitems}
\item $-$ \textsl{Scrum Master and Software Engineer:} Developer and Software Architect of several applications used in
	the ESTRACK Ground Stations in line with the European Space Ground Operations Software, such as EUDGSSC (a Rich Client
	Application for interacting with the Subsystem Controllers) or the EUD MIMICS (a mimic platform to provide a graphical
	environment for the EGOS Applications).
\end{pitems}}
\ccventry{}{}{}{}{
\begin{pitems}
\item $-$ \textsl{Validation Engineer:} Definition and application of validation procedures and documentation for several 
	projects deployed on ESA Ground Stations like FIDES (File Disseminator for ESTRACK Ground Stations) or 
	FARC (File Archive used as SCOS telemetry storage).
\item $-$ \textsl{TMTCS Engineer:} Technical responsible for the Telemetry and Telecommand System maintenance. In charge of the
	continous validation and integration of the system. Creation of the appropiate SPRs, SCRs and Technical Notes related
	to the day-to-day work by using several tools, such as DOORS, Mercury Quality Center and Synergy Database. Participation in
	the Radio Frequency Compatibility Test (RFCT) on the TMTCS side for several missions, such as Yinghuo or Fobos-Grunt.
    Adaptation of the TMTCS to the ESA Mission Gaia for achieving the highly demanding telemetry rates on the Ground Station
	Infrastructure.
\item $-$ \textsl{Server Administrator:} System administrator of the Ground Infrastructure Back-end Server used for
	active development activities and documentation. Setup of the system using Continuous Integration tools, like SVN,
	Hudson, Trac, Ant, IceScrum among others.
\item $-$ \textsl{Technical Consultor:} Participation in Technical Comitees and Task Forces as Advisor in many internal
	projects and activities, such as FEC Evolution, STC2 and GSMC.
\end{pitems}}
\ccventry{2008--2009}{YGT at ESOC/ESA}{Young Graduate Trainee at ESOC/ESA OPG-GIB Section}{Darmstadt, Germany}
{. Activities:
\begin{pitems}
\item $-$ \textsl{Validation Engineer:} Technical advisor and supervisor of the Factory Acceptance Tests for the
	Egos User Desktop and Technical manager for the Site Acceptance Tests. Validation of FARC from S2K for its use
	with FIDES under the EGOS Model. 
\item $-$ \textsl{Software Engineer:} Technical revision for the requirements of the MIMIC project and development 
	of an internal prototype based on Eclipse using EMF, GEF and GMF.
\item $-$ \textsl{Documentation: } Creation of several Technical Notes, SRSs, ICDs, MoMs and other internal documents.
\end{pitems}}
\cventry{2007--2007}{Teacher of Project Management}{Project management seminars provided to IMERYS IKF Inc.}{A Guarda, Pontevedra, Spain}
{Enterprise course teacher for Norformacion - Fundación Tripartita to IMERYS IKF Inc. about Project Management theorical aspects and the use
of the Microsoft Project for designing Gantt Diagrams and keeping track of the project changes}{}
\cventry{2005--2006}{Contracted Researcher}{Vigo University, Spain}{Development of JPoEML, 
	a Java tool (mimic oriented) for designing and producing an E-Learning web environment that uses XML and J2EE to serve
	the desired learning models to the users over the web. This tool, created from scratch and based on the popular JFreeChart 
	Java Library, presented an editing environment similar to a UML Designing Tool used for creating and editing Learning Environments}{}{}
\cventry{1999--2005}{Network Administrator and Web Developer}{Freelance web developer and network administrator for local 
	and national enterprises. Design and deployment of a set of VPNs to provide media content to final clients through a Web
	Environment}{Vigo, Spain}{}{}

\pagebreak

\section{Professional Training}
\cventry{Apr 2010}{Agile Development Technologies}{The Server Labs}{ESOC, Darmstadt, Germany}{Software Development using Agile Metodology and SCRUM Techniques}{}
\cventry{Feb 2009}{Eclipse EMF, GEF and GMF Technologies}{OBEO Eclipse Foundation Member}{ESOC, Darmstadt, Germany}{Development of Eclipse Applications based on EMF/GEF and GMF}{}{}


\section{Languages}
\cvlanguage{Spanish}{Mother Tongue}{}
\cvlanguage{English}{Fluent}{Three years working in English at the European Space Agency. Six months living in Toronto, Canada.}
\cvlanguage{German}{Good}{Three years living in Germany, Goethe Institut level B2.}
\cvlanguage{Italian}{Basic}{Internal courses at ESOC/ESA.}

\section{Software Development}
\cvskill{Linear}{Assembler, C, Pascal, Cobol, Fortran, Lisp.}
\cvskill{Object Oriented}{Java, C++, Delphi, Visual Basic.}
\cvskill{Eclipse}{PDE, RCP, SWT, EMF, GEF, GMF, OCL.}
\cvskill{J2EE}{JSF, JPA, Hibernate, Richfaces.}
\cvskill{Distributed}{RCP, Java RMI, CORBA.}
\cvskill{Scripting}{Python, Bash, Tcl, Perl.}
\cvskill{Web}{PHP, ASP, JSP, GWT, DHTML, XML, CSS, JS, AJAX.}
\cvskill{CI}{Hudson, Jira, SVN, Trac, Maven, Ant.}
\cvskill{Agile}{Agilefant, IceScrum, XPlanner.}
\cvskill{Others}{SQL, \TeX.}

\section{Computer Environments}
\cvline{Unix}{Debian and RedHat Linux, Solaris and Irix System Administration. LAMP.}
\cvline{Windows}{NT/XP/2003/7 System Administration.}
\cvline{Networking}{HTTP, DNS, SMTP, FTP, Samba, DHCP, SSH, NFS, Security.}
\cvline{Virtualization}{Xen, OpenVZ, VirtualBox, VMWare.}
\cvline{Cloud}{Amazon EC2 S3 AutoScaling ELB Route53, Google Apps.}

\section{Documentation and References}
\cvline{}{\slshape{Available upon Request.}}

\end{document}


%% end of file `template_en.tex'.
